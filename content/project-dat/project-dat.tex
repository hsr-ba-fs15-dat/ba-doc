\part{Project DAT}

\chapter{Einführung}

In der heutigen Gesellschaft wird der Austausch von Daten immer wichtiger. Vermehrt bieten werden grosse Datenmengen unterschiedlichster Art - aus Forschung, Regierung oder zivilen Kreisen - zur allgemeinen Verwendung angeboten. Nicht nur die Art der Daten ist jedoch vielfältig, sondern auch Datenformat oder Zugriffsart unterscheiden sich.

Das \gls{dat}-Projekt will die Integration verschiedener solcher Datenquellen vereinfachen.

\section{Was ist dat?} % https://github.com/maxogden/dat/blob/master/docs/what-is-dat.md

Das dat-Projekt hat folgende Ziele: 

\begin{itemize}
\item Daten sollen automatisch zwischen unterschiedlichen dat-Instanzen synchronisiert werden können. 
\item Unterstützung grosser Datenmengen (Milliarden von Datensätzen bzw. Speicherbedarf im TB-Bereich), evtl. mit häufigen Aktualisierungen
\item Unterstützung von Daten in Tabellenform oder unstrukturiert
\item Plugin-basierte Schnittstelle zu bestehenden Datenbanken/formaten
\item Unterstützung für automatisierte Workflows
\end{itemize}

\section{Architektur}
% diagramm, requirements, interfaces, ...

\xxx[Architektur-doc finden]

\section{Use Cases} % wofür zum teufel ist das teil gedacht. einige use cass sind online beschrieben

\subsection{Astronomie: Trillian} % https://github.com/maxogden/dat/issues/172, https://github.com/trillian/trillian
In der Astronomie fallen riesige Datenmengen an. Teilweise werden diese als grosse Daten-Releases zur Verfügung gestellt (z.B. Sloan Digital Sky Survey), welche frühere Releases komplett ersetzen. Andere Projekte stellen inkrementelle Updates zur Verfügung (z.B. Hubble). Viele Astronomie-Institute haben weder die Mittel noch das Know-How, mit solchen Datenmengen umgehen zu können. Das Trillian-Projekt kümmert sich um die Verwaltung dieser Daten und bietet eine Compute-Engine an, um neue Modelle anhand der vorhandenen Daten zu testen.

dat soll für den Import und die Indexierung der Daten verwendet werden.

\subsection{Regierung: Sammlung von Daten} % https://github.com/maxogden/dat/issues/153
Für Statistik- oder Regulierungszwecke sammeln Regierungen Daten von unterschiedlichsten Betrieben. Diese Daten müssen oft über ein mehr oder weniger brauchbares Portal abgeliefert werden.

Die Verwendung von dat bringt folgende Vorteile:
\begin{itemize}
\item Ermöglicht die Prüfung von Daten beim Import, bereits vor der Ablieferung der Daten
\item Ermöglicht Ergänzung oder Korrektur von bereits abgelieferten Daten.
\item Ersatz der bisher verwendeten, für jede Regierungsstelle neu kreierten und teuren, Portale durch eine standardisierte Lösung.
\end{itemize}

\chapter{Erweiterung}

\section{Datskript}

\section{Gasket}

%https://github.com/datproject/datscript

\section{Schnittstellen / API}

\subsection{REST}
%https://github.com/maxogden/dat/blob/master/docs/rest-api.md

\subsection{JavaScript}
%https://github.com/maxogden/dat/blob/master/docs/js-api.md

\subsection{CLI}
%https://github.com/maxogden/dat/blob/master/docs/cli-usage.md

\subsection{Python}
%https://github.com/pkafei/Dat-Python

