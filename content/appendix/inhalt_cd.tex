\chapter{Inhalt der CD}

%, rounded corners
\tikzstyle{every node}=[ultra thick,anchor=west, font={\scriptsize\ttfamily}, inner sep=4pt]

\tikzstyle{selected}=[ultra thick, draw=black]
\tikzstyle{thisdoc}=[ultra thick, selected, draw=red!90!black, fill=red!30]
\tikzstyle{comment}=[ultra thick, selected, font={\scriptsize}]
\tikzstyle{root}=[ultra thick, selected, fill=gray!30]

\begin{figure}[H]
	\centering
	\begin{tikzpicture}[
	%scale=.7,
	grow via three points={one child at (1,-0.7) and
		two children at (1,-0.7) and (1,-1.4)},
	edge from parent path={(\tikzparentnode.south) |- (\tikzchildnode.west)}]
	
	\node [root] {CD}
	child { node [thisdoc] {ba-opendatahub\_chuesler\_rliebi\_fscala.pdf {\normalfont$\rightarrow$ Dieses Dokument}}}
	child { node [selected] {0\_INFO}
		child { node [comment] {Diverse Info- und Best-Practice Dokumente für die \acs{ba}}}
	}
	child [missing] {}
	child { node [selected] {1\_AUFGABE}
		child { node [comment] {Original Aufgabenstellung}}
	}
	child [missing] {}
	child { node [selected] {2\_DOC}
		child { node [comment] {Quellen der \LaTeX\ Dokumentation inkl. Sitzungsprotokollen}}
		child { node [selected] {main.pdf}}
	}
	child [missing] {}
	child [missing] {}
	child { node [selected] {3\_SRC}
		child { node [selected] {vm}}
		child { node [selected] {opendatahub}}
		child { node [selected] {heroku-buildpack-scipy}}
		child { node [selected] {\dots}
			child { node [comment] {Weitere Git Repositories der GitHub Organisation hsr-ba-fs15-dat}}
		}
	}
	child [missing] {}
	child [missing] {}
	child [missing] {}
	child [missing] {}
	child [missing] {}
	child { node [selected] {4\_DIST}
		child { node [selected] {app}
			child { node [comment] {Deploybare Applikation}}
		}child [missing] {}
		child { node [selected] {doc}
			child { node [comment] {Generierte Sphinx/TSDoc Dokumentation des Quellcodes}}
		}
	}
	child [missing] {}
	child [missing] {}
	child [missing] {}
	child [missing] {}
	child { node [selected] {5\_BROSCHUERE}}
	child { node [selected] {6\_POSTER}}
	child { node [selected] {7\_MISC}
		child { node [comment] {Atlassian Jira Backup, Time-Tracking Export, \dots}}
	}
	child [missing] {};
	\end{tikzpicture}
	\caption{Inhalte der beigefügten CD}

\label{fig:inhalt-cd}
\end{figure}