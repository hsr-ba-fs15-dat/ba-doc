\chapter{dat CLI Help}\label{app:dat-help}

\begin{src}{console}
$ dat help

Example usage

make a new folder and turn it into a dat store

    mkdir foo
    cd foo
    dat init

put a JSON object into dat

    echo '{"hello": "world"}' | dat import --json

stream the most recent of all rows

    dat cat

stream a CSV into dat

    cat some_csv.csv | dat import --csv

or

    dat import --csv some_csv.csv

use a custom newline delimiter:

    cat some_csv.csv | dat import --csv --newline $'\r\n'

use a custom value separator:

    cat some_tsv.tsv | dat import --csv --separator $'\t'

stream NDJSON into dat

You can pipeline Newline Delimited JSON (
NDJSON (http://ndjson.org/)
) into dat on stdin and it will be stored

    cat foo.ndjson | dat import --json

specify a primary key to use

    echo $'a,b,c\n1,2,3' | dat import --csv --primary=a
    echo $'{"foo":"bar"}' | dat import --json --primary=foo

attach a blob to a row

    dat blobs put jingles jingles-cat-photo-01.png

stream a blob from a row

    dat blobs get jingles jingles-cat-photo-01.png

add a row from a JSON file

    dat rows put burrito-recipe.json

get a single row by key

    dat rows get burrito

delete a single row by key

    dat rows delete burrito

but rows are never truly deleted. you can always go find a row at the version it was last seen. in this case, that row was at version 1

    dat rows get burrito 1

start a dat server

    dat listen

then you can poke around at the REST API:

    /api/changes
    /api/changes?data=true
    /api/metadata
    /api/rows/:docid
    POST /api/bulk content-type: application/json (newline separated json)

pull data from another dat

    dat pull http://localhost:6461

push data to another dat

    dat push http://localhost:6461

delete the dat folder (removes all data + history)

    dat clean

view raw data in the store

    npm install superlevel -g
    superlevel .dat/store.dat createReadStream
\end{src}