%------------------------------------------------------------------------------
% Glossary
%------------------------------------------------------------------------------

\newglossaryentry{osm2pgsql}{name=osm2pgsql, description={ist ein Tool um OSM Daten in eine PogstreSQL PostGIS Datenbank zu importieren},url={http://wiki.openstreetmap.org/wiki/Osm2pgsql}}

\newglossaryentry{psql}{name=PostgreSQL, description={ist ein freies, objektrelationales Datenbankmanagementsystem},url={http://www.postgresql.org/}}

\newglossaryentry{pgis}{name=PostGIS, description={ist eine PostgreSQL Erweiterung welche geographische Funktionalität und Objekte zur Verfügung stellt um beispielsweise Ortsabfragen zu ermöglichen},url={http://postgis.net/}}

\newglossaryentry{osmosis}{name=Osmosis, description={ist ein Tool um OpenStreeMap Kartendaten zu manipulieren (packen/entpacken, Ausschnitte extrahieren, \dots)},url={https://github.com/openstreetmap/osmosis}}

\newglossaryentry{mopublic}{name=MOpublic, description={Komplexes Datenschema für Gebäudeadressen},url={http://www.cadastre.ch/internet/cadastre/en/home/products/mopublic.html}}

\newglossaryentry{dat}{name=dat, description={Open-Source Plattform für tabellarische Daten mit ähnlichem CLI wie Git},url={http://dat-data.com/}}


\newglossaryentry{sloan-foundation}{name=Sloan Foundation, description={ist eine amerikanische non-profit Stiftung für Forschung in den Bereichen Naturwissenschaften und Technolgien},url={http://www.sloan.org/}}


\newglossaryentry{knight-foundation}{name=Knight Foundation, description={ist eine non-profit Stiftung für diverse Zwecke},url={http://www.knightfoundation.org/}}

\newglossaryentry{kettle}{name=Pentaho Kettle, description={ist eine Datenintegrations/ETL-Software von Pentaho},url={http://community.pentaho.com/projects/data-integration/}}

\newglossaryentry{interlis}{name=INTERLIS, description={erlaubt die Beschreibung von Datenmodellen sowie den Austausch von (Geo-)Daten}, url={http://www.interlis.ch/}}

\newglossaryentry{geocsv}{name=GeoCSV, description={ist ein durch die das Geometa Lab der HSR spezifizierte Erweiterung von CSV mit Geo-Daten}, url={http://giswiki.hsr.ch/GeoCSV}}


\newglossaryentry{geojson}{name=GeoCSV, description={ist eine auf JSON basierende Schema-Spezifikation für Geo-Daten}, url={http://geojson.org/}}


%------------------------------------------------------------------------------
% Acronyms
%------------------------------------------------------------------------------
\newacronym
[description={\acrlong{mit}. Die \acrshort{mit} Lizenz, welche aus dem \acrshort{mit} stammt, erlaubt die Wiederverwendung des Programmcodes sowohl für andere offene aber auch nicht offene bzw. kommerzielle Software},url={http://opensource.org/licenses/MIT}]
{mit}{MIT}{Massachusetts Institute of Technology}

\newacronym
[description={\acrfull{wsgi} ist ein Python Standard welches definiert wie ein Webserver wie z.B. Apache mit der Applikation kommunizieren soll},url={http://wsgi.readthedocs.org/}]
{wsgi}{WSGI}{Web Server Gateway Interface}


\newacronym{scm}{SCM}{Source Control Management}

\newacronym{avt}{AVT}{Arbeiten Verwaltungs Tool der \acrshort{hsr}}

\newacronym
[description={\acrlong{osm} ist ein Projekt, welches open source geografische Daten bereitstellt},url={http://www.openstreetmap.org/}]
{osm}{OSM}{OpenStreetMap}

\newacronym{ba}{BA}{Bachelorarbeit}

\newacronym
[url={http://giswiki.hsr.ch/EOSMDBOne}]
{eosmdbone}{EOSMDBOne}{Enhanced OpenStreetMap Database One}


\newacronym
[description={\acrlong{pep8} definiert die Coding Konventionen bzw. Style Guidelines für sämtlichen Code der Python Standard Library},url={https://www.python.org/dev/peps/pep-0008/}]
{pep8}{PEP~8}{Python Enhancement Proposal 8}

\newacronym{hcid}{HCID}{Human Computer Interaction and Design}


\newacronym
[url={http://hsr.ch/}]
{hsr}{HSR}{Hochschule für Technik Rapperswil}


\newacronym
[url={http://www.ifs.hsr.ch/}]
{ifs}{IFS}{Institut für Software, ein Institut der \acs{hsr}}

\newacronym{crud}{CRUD}{Create-Update-Delete}

\newacronym
[description={\acrlong{xml} ist eine deskriptive Sprache für hierarchische Daten}]
{xml}{XML}{Extensible Markup Language}


\newacronym{ci}{CI}{Continuous Integration}

\newacronym[shortplural=NFRs,longplural=Nicht-funktionale Anforderungen]{nfr}{NFR}{Nicht-funktionale Anforderung}

\newacronym{orm}{ORM}{Object-relational Mapping}

\newacronym{i18n}{I18N}{Internationalisierung}

\newacronym{l10n}{L10N}{Lokalisierung}

\newacronym
[description={\acrlong{yaml} ist eine beschreibende Sprache für Daten welche aufgrund dessen Lesbarkeit oft für Konfigurationsdaten verwendet wird},url={http://www.yaml.org/}]
{yaml}{YAML}{YAML Ain't Markup Language}


\newacronym{gis}{GIS}{Geographic information system} % (Geoinformationssystem) XXX

\newacronym
[description={''\acrlong{yagni}'' ist ein Prinzip aus der agilen Softwareentwicklung und besagt, dass keine Funktionalität implementiert werden soll, bis diese gebraucht wird}]
{yagni}{YAGNI}{You Ain't Gonna Neet It}

\newacronym
[description={\acrlong{poi} (dt. Ort von Interesse)},shortplural=POIs,longplural=Points of Interest]
{poi}{POI}{Point of Interest}

\newacronym
[longplural={Single Page Applications},description={\acrlongpl{spa} sind Webapplikationen, dessen HTML nur zu Beginn geladen wird und danach nur noch via API mit dem Server kommuniziert}]
{spa}{SPA}{Single Page Application}

\newacronym
[description={\acrlong{sql}. Eine deskriptive Sprache zur Formulierung von Abfragen aus Datenbanken}]
{sql}{SQL}{Structured Query Language}

\newacronym
[shortplural=CMS',longplural=Content Management Systeme]
{cms}{CMS}{Content Management System}

\newacronym
[description={\acrlong{rest}}]
{rest}{REST}{Representational State Transfer}

\newacronym
[description={\acrlong{api}}]
{api}{API}{Application Programming Interface}

\newacronym
[url={https://pypi.python.org/},description={\acrlong{pypi}. Zentrales Repository für Python Module.}]
{pypi}{PyPI}{Python Package Index}

\newacronym
[description={\acrlong{csv}. Textdatei mit einem Datensatz pro Zeile wobei die einzelnen Felder durch Kommas getrennt werden}]
{csv}{CSV}{Comma Separated Values}

\newacronym
[description={\acrlong{wfs}. OGC Standard Webservice für Geo-Daten},url={http://www.opengeospatial.org/standards/wfs}]
{wfs}{WFS}{Web Feature Service}

\newacronym
[description={\acrlong{dsl}. Domänenspezisiche Sprache, speziell Entwickelt für eine bestimmte Domäne bzw. Problemstellung}]
{dsl}{DSL}{Domain Specific Language}

\newacronym
[description={\acrlong{ndjson}. Eignet sich durch die zeilenweise Trennung besonders für Pipelining},url={http://ndjson.org/}]
{ndjson}{NDJSON}{Newline delimited JSON}

\newacronym
[description={\acrlong{trobdb}. Ist eine Webapplikation für Verkehrshindernisse der Schweiz},url={http://trobdb.hsr.ch/}]
{trobdb}{TROBDB}{Traffic Obstruction Database}

\newacronym
[description={\acrlong{saga.ch}},url={http://www.ech.ch/vechweb/page?p=dossier&documentNumber=eCH-0014}]
{saga.ch}{SAGA.ch}{Standards und Architekturen für eGovernment Anwendungen Schweiz}

\newacronym
[description={\acrlong{cli}. Kommandozeilen-Tool}]
{cli}{CLI}{Command-line interface}

\newacronym{odh}{ODH}{OpenDataHub}

\newacronym
[description={\acrlong{odhql}. Transformationssprache für \acs{odh} basierend auf \acs{sql}}]
{odhql}{ODHQL}{\acl{odh} Query Language}

\newacronym
[description={\acrlong{etl}. Bezeichnung für den Prozess bei welchem heterogene Daten unterschiedlicher Herkunft als homogene Daten an einem zentralen Ort vereinigt werden}]
{etl}{ETL}{Extract, Transform, Load}

\newacronym
[description={\acrlong{kml}. Ein Dateiformat zum Austausch von Geodaten. KML wurde durch die Verwendung in Google Earth bekannt.}]
{kml}{KML}{Keyhole Markup Language}

\newacronym
[description={\acrlong{gml}. Ein Dateiformat zum Austausch von Geodaten}]
{gml}{GML}{Geometry Markup Language}

\newacronym
[description={\acrlong{gdal}. Unterliegende Bibiothek bei fast allen Geo-Applikationen und Tools (ogr2ogr, shapely, etc.)}]
{gdal}{GDAL}{Geospatial Data Abstraction Library}

\newacronym
[description={\acrlong{fme}. Eine proprietäre und kommerzielle Desktop-Applikation zur Erstellung und Ausführung von ETL Prozessen mittels grafischer Oberfläche},url={http://www.safe.com/}]
{fme}{FME}{Feature Manipulation Engine}

\newacronym{JWT}{JWT}{JSON Web Token}
\newacronym
{vm}{VM}{Virtuelle Maschine}

\newacronym
[description={\acrlong{wkt}. Textuelle, Menschen-lesbare Repräsentation von Geometrien.},url={http://www.geoapi.org/3.0/javadoc/org/opengis/referencing/doc-files/WKT.html}]
{wkt}{WKT}{Well-Known Text}


\newacronym
[description={\acrlong{ogc}. Gemeinnützige Organisation mit dem Ziel der Festlegung von Standards im Geo-Daten Bereich.},url={http://www.opengeospatial.org/}]
{ogc}{OGC}{Open Geospatial Consortium}

\newacronym
[description={\acrlong{ckan}. Web-basiertes System zum Speichern und Verteilen von Daten.}]
{ckan}{CKAN}{Comprehensive Knowledge Archive Network}

\newacronym
[description={\acrlong{mvc}. Der englischsprachige Begriff model view controller (MVC, englisch für Modell-Präsentation-Steuerung) ist ein Muster zur Strukturierung von Software-Entwicklung in die drei Einheiten Datenmodell (engl. model), Präsentation (engl. view) und Programmsteuerung (engl. controller).}]
{mvc}{MVC}{Model, View, Controller}

\newacronym
[description={\acrlong{mvvm}. Model View ViewModel (MVVM) ist eine Variante des Model View Controller-Musters (\acs{mvc}) zur Trennung von Darstellung und Logik der Benutzerschnittstelle (UI). Es zielt auf moderne UI-Plattformen wie Windows Presentation Foundation (WPF), JavaFX, Silverlight und HTML5 ab.}]
{mvvm}{MVVM}{Model, View, ViewModel}

\newacronym
[description={Vagrant ist eine auf Ruby basierte Open Source-Anwendung, die das Verwalten und Erstellen von virtuellen Maschinen ermöglicht. Vagrant dient eigentlich als Wrapper zwischen der Virtualisierungssoftware (VirtualBox, VMware, Parallels, etc.) und Systemkonfiguartionswerkzeugen (in unserem Fall Puppet).
Der grosse Mehrwert von Vagrant ergibt sich dadurch, dass es komplett Plattform- und Programmiersprachenunabhängig ist und somit für verschiedene Softwareprojekte verwendet werden kann. Im Vagrantfile wird eine virtuelle Maschine definiert und konfiguriert. Dieses File wird mit dem Projekt in der Versonskontrolle abgelegt, so kann auf den unterschiedlichen Host-Systemen gewährleistet werden, dass mit der selben Entwicklungsumgebung gearbeitet werden kann und so alle Abhängigkeiten, ohne das Hostsystem zu beeinflussen, installiert werden. }, url={http://vagrantup.com}]
{vagrant}{Vagrant}{Vagrant}

\newacronym
[description={\acrlong{PyCharm}. PyCharm ist eine Integrierte Entwicklungsumgebung (IDE) der Firma JetBrains für die Programmiersprache Python.}]
{PyCharm}{PyCharm}{Model, View, ViewModel}

\newacronym
[description={TypeScript ist eine vom Unternehmen Microsoft entwickelte Typeen Orientierte Programmiersprache, die auf den Vorschlägen zum zukünftigen ECMAScript-6-Standard(JavaScript) basiert.}, url={http://www.typescriptlang.org}]
{ts}{TS}{TypeScript}

\newacronym
[description={Auch als Speicherschutzverletzung bekannt. Dieser Fehler tritt vor allem auf, wenn versucht wird, auf geschützte Speicherbereiche zuzugreifen. Siehe auch \url{https://de.wikipedia.org/wiki/Schutzverletzung}.}]
{segfault}{segfault}{Segmentation Fault}