\section{Sprint 0}

\subsection{Summary}

\begin{table}[H]
	\centering
	\begin{tabular}{ll}
		\toprule
		\multicolumn{2}{c}{\textbf{Sprint 0} \textit{(Kickoff)}}\\
		\midrule
		\textbf{Periode} & 16.02.2015\textendash 22.02.2015\\
		\textbf{Stunden Soll} & \SI{48}{\hour}\\
		\textbf{Stunden Plan} & \textendash \\
		\textbf{Stunden Ist} & \SI{42.25}{\hour}\\
		\bottomrule
	\end{tabular}	
\end{table}


\subsection{Ziele}
\begin{itemize}
	\item Kickoff
	\item Aufgabenstellung verstehen
	\item Infrastruktur aufsetzen
\end{itemize}


\subsection{Abgeschlossen}
Folgende High-level (ohne Subtasks) Tasks wurden während des ersten Sprints erstellt und sogleich abgeschlossen.

\begin{table}[H]
\centering
\begin{tabular}{ll}
	\toprule
	\textbf{JIRA-Key} & \textbf{Summary}\\
	\midrule
		DAT-2 &	Infrastruktur aufsetzen	Task\\
		DAT-3 &	Dokumentation aufsetzen	Task\\
		DAT-4 &	Projektmeeting Woche 1 (Kickoff)\\
		DAT-15 & Grobe Projektplanung\\
		DAT-16 & Einlesen dat\\
	\bottomrule
\end{tabular}	
\end{table}

\subsection{Probleme}
Aufgrund einer noch fehlenden Note war die Beteiligung von \chuf an diesem Projekt ungewiss. Diese Zweifel konnten rasch geklärt werden und das Team ist froh ihn bei der Bachelorarbeit dabei zu haben.