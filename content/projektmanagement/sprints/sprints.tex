\chapter{Sprints}

\subimport{}{sprint-0.tex}
\subimport{}{sprint-1.tex}
\subimport{}{sprint-2.tex}
\subimport{}{sprint-3.tex}
\subimport{}{sprint-4.tex}
\subimport{}{sprint-5.tex}
\subimport{}{sprint-6.tex}
\subimport{}{sprint-7.tex}
\subimport{}{sprint-8.tex}


\section{Total und Fazit}

\begin{table}[H]
	\centering
	\begin{tabular}{ll}
		\toprule
		\multicolumn{2}{l}{\textbf{Total}}\\
		\midrule
		\textbf{Periode} & 16.02.2015\textendash 12.06.2015\\
		\textbf{Stunden Soll} & \SI{960}{\hour}\\
		\textbf{Stunden Ist} & \SI{1404.3}{\hour}\\
		\bottomrule
	\end{tabular}
\end{table}

\begin{table}[H]
	\centering
	\begin{tabular}{ll}
		\toprule
		\multicolumn{2}{l}{\textbf{Total pro Person}}\\
		\midrule
		\textbf{Remo Liebi} & \SI{491}{\hour}\\
		\textbf{Christoph Hüsler} & \SI{436.5}{\hour}\\
		\textbf{Fabio Scala} & \SI{476.8}{\hour}\\
		\bottomrule
	\end{tabular}	
\end{table}

Wie in den Burndown Charts und Zeitauswertung der jeweiligen Sprints ersichtlich ist, wurde der Aufwand relativ gut geschätzt. Weniger gut waren die Schätzungen einzelner Tasks, was in den Charts durch die Differenz zwischen verbleibendem Aufwand und verbuchten Aufwand dargestellt wird.

