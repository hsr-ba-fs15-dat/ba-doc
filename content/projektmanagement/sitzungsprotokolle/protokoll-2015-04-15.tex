\documentclass[class=scrbook,crop=false]{standalone}

\input{./preamble}

\begin{document}
	
	\section{Projektsitzung 15. April 2015}
	
	\begin{tabular}{ll}
		\textbf{Datum} & 15.4.2015 \\
		\textbf{Zeit} & 14:00\textendash15:30 Uhr \\
        \textbf{Ort} & \acs{ifs} an der \acs{hsr} \\
        \textbf{Anwesende} & \proff \\ & \chuf \\ & \rlif \\ & \fscf 
	\end{tabular}

	\subsection*{Traktanden}
	\begin{itemize}
		\item Aktuellen Stand besprechen
		\item Probleme mit Interlis 2
		\item UI besprechen
		\item Scope DSL-Syntax (anhand Vergleich mit Pentaho Kettle)
	\end{itemize}
	
	\subsection*{Was wurde erreicht}
	\begin{itemize}
		\item Dokumentation der diversen Optionen (Transformationssprache)
		\item Erstes UI für Transformation
		\item Support für Interlis 1
		\item Interpreter auf dem selben Stand wie der Parser
	\end{itemize}

	\subsection*{Beschlüsse}
	\begin{itemize}
		\item Begriff ``Dokumente'' ersetzen da missverständlich. CKAN verwendet ``Package''.
		\item Interlis 2 wird vorerst nicht unterstützt.
	\end{itemize}
	
	\subsubsection*{Weiteres Vorgehen}
	\begin{itemize}
		\item Implementation von ``kaskadierenden'' Online-Quellen (Auto-Update)
		\item Persistente Transformationen
		\item Caching
	\end{itemize}
\end{document}
