\documentclass[class=scrbook,crop=false]{standalone}

\input{./preamble}

\begin{document}
	
	\section{Projektsitzung 1. April 2015}
	
	\begin{tabular}{ll}
		\textbf{Datum} & 1.4.2015 \\
		\textbf{Zeit} & 10:00\textendash11:00 Uhr \\
        \textbf{Ort} & \acs{ifs} an der \acs{hsr} \\
        \textbf{Anwesende} & \proff \\ & \chuf \\ & \rlif \\ & \fscf
	\end{tabular}
	
	\subsection*{Traktanden}
	\begin{itemize}
		\item Aktuellen Stand besprechen bzw. Demo
		\item SuisseID
		\item Beschluss Transformationssprache ``OQL => OpenDataHub Query Language''
	\end{itemize}
	
	\subsection*{Was wurde erreicht}
	\begin{itemize}
		\item Fixes \& Refactorings
		\item Erster Parser für OQL
	\end{itemize}
	
	\subsection*{Beschlüsse}
	\begin{itemize}
		\item Es wird nur 1 Geometrie pro Datei erlaubt.
		\item Es wird OpenID und SuisseID zusätzlich implementiert
		\item Die Transformationssprache heisst ODHQL und nicht wie bisher OQL um Verwechslung mit diversen Anderen Query Languages zu vermeiden.
		\item Reserved Keywords für ODHQL müssen definiert werden
	\end{itemize}
	
	\subsubsection*{Weiteres Vorgehen}
	\begin{itemize}
		\item SuisseID \& OpenID Authentifizierung
		\item Datentypen und Schema überlegen
	\end{itemize}
\end{document}
