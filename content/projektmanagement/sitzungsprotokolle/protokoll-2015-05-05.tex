\documentclass[class=scrbook,crop=false]{standalone}

\input{./preamble}

\begin{document}
	
	\section{Projektsitzung 5./6. Mai 2015}
	
	\begin{tabular}{ll}
		\textbf{Datum/Zeit} & 5.5.2015 15:00\textendash16:00 Uhr \& 6.5.2015 14:30\textendash15:00 Uhr \\
        \textbf{Ort} & \acs{ifs} an der \acs{hsr} \\
        \textbf{Anwesende} & \proff \\ & \chuf \\ & \rlif \\ & \fscf 
	\end{tabular}

	\subsection*{Traktanden}
	\begin{itemize}
		\item Aktuellen Stand mit Demo \& Feedback
		\item Fragen bzgl. GeoCSV Format
		\item Letzte Möglichkeit für ``must'' Features
		\item Wenn Zeit: Stand Dokumentation (evtl. nur Inhaltsverzeichnis) besprechen.
	\end{itemize}
	
	\subsection*{Was wurde erreicht}
	\begin{itemize}
		\item Transformationen für alle TROBDB Datenquellen
		\item Nested ("kaskadierende") Transformationen (Transformation als Input für Transformation)
		\item Refactoring Transformations-GUI
		\item Generierung ODHQL Dokumentation/Referenz
		\item Diverse kleine Verbesserungen
		\item Neu wird KML als Zwischenformat für ogr2ogr verwenden
	\end{itemize}

	\subsection*{Beschlüsse}
	\begin{itemize}
		\item GeoCSV Unterstützung als letztes Feature dieses Sprints
		\item UTF-8 für GeoCSV verwenden
		\item Nur GitHub und Facebook als oAuth Provider
	\end{itemize}
	
	\subsubsection*{Weiteres Vorgehen}
	\begin{itemize}
		\item Letzter Feature-Sprint bis 11.05
		\begin{itemize}
			\item \gls{interlis} Schema-Generierung
			\item Evtl. GeoCSV Unterstützung
		\end{itemize}
		\item Danach folgt ein Bugfixing Sprint
	\end{itemize}
\end{document}
