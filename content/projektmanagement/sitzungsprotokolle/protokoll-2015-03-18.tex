\documentclass[class=scrbook,crop=false]{standalone}

\input{./preamble}

\begin{document}
    
    \section{Projektsitzung 18. März 2015}
    
    \begin{tabular}{ll}
        \textbf{Datum} & 18.03.2015 \\
        \textbf{Zeit} & 09:30\textendash10:30 Uhr \\
        \textbf{Ort} & \acs{ifs} an der \acs{hsr} \\
        \textbf{Anwesende} & \proff \\ & \chuf \\ & \rlif \\ & \fscf
    \end{tabular}

    \subsection*{Traktanden}
    \begin{itemize}
        \item Aktuellen Stand besprechen
        \item Hindernisse besprechen
    \end{itemize}
    
    \subsection*{Was wurde erreicht}
    \begin{itemize}
        \item OAuth-Einbindung begonnen (facebook, GitHub)
        \item Layoutanpassung
        \item REST API erweitert
        \item Dokumentenliste
    \end{itemize}
    
    \subsection*{Beschlüsse}
    \begin{itemize}
        \item Die verschiedenen Varianten zur Daten-Speicherun (File, DB/Record, DB/Schemalos) sollen verglichen werden.
        \item Authentifizierung: SuisseID soll ebenfalls unterstützt werden
        \item Ein Interface für interaktives Schema-Mapping ist nicht Bestandteil der Aufgabe. Stattdessen sollen von Entwicklern zur Verfügung gestellte Transformationen verwendet werden.
    \end{itemize}
    
    \subsubsection*{Weiteres Vorgehen}
    \begin{itemize}
        \item Refactoring
        \item Entscheidung und evtl. Umbau Datenmodell
        \item Einbindung PETL/Pandas/OGR abklären
        \item OAuth-Einbindung fertigstellen
    \end{itemize}
    
\end{document}
