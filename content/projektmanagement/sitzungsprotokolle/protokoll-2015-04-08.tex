\documentclass[class=scrbook,crop=false]{standalone}

\input{./preamble}

\begin{document}
	
	\section{Projektsitzung 8. April 2015}
	
	\begin{tabular}{ll}
		\textbf{Datum} & 8.4.2015 \\
		\textbf{Zeit} & 9:30\textendash11:30 Uhr \\
        \textbf{Ort} & \acs{ifs} an der \acs{hsr} \\
        \textbf{Anwesende} & \proff \\ & \mstolze (10:00 - 11:00 Uhr) \\ & \chuf \\ & \rlif \\ & \fscf 
	\end{tabular}
	
	\subsection*{Traktanden}
	\begin{itemize}
		\item Aktuellen Stand besprechen
		\item Präsentation, Demo mit \mstolze
	\end{itemize}
	
	\subsection*{Was wurde erreicht}
	\begin{itemize}
		\item Parser (mehr oder weniger vollständig) und Interpreter (work in progress) für unsere Transformations-Sprache \acs{odhql}. Die Sprache ist durch eigene Funktionen erweiterbar. Bereits implementiert sind einige String- und Geometriefunktionen.
		\item Erstes UI für Transformationen
	\end{itemize}

	\subsection*{Beschlüsse}
	\begin{itemize}
		\item Erstellen eines Domänen-spezifischen Gloassars (Transformer, Quellen, Senken, etc.)
		\item Transformationen sollen als Input für andere Transformationen dienen und gruppiert werden können (analog Dokumente/FileGroups)
	\end{itemize}
	
	\subsubsection*{Weiteres Vorgehen}
	\begin{itemize}
		\item \prof liefert Inputdaten für UC Gebäudeadressen, NFRs vs. Machbarkeitsstudie, Zugang zu Source-Code von geoconverter.hsr.ch
		\item Dokumentation: NFRs, Motivation (Inter-Organisations-Kommunikation, evtl. Presseberichte), Abgrenzung Konkurrenz
		\item Interpreter und UI für Transformationen erweitern/finalisieren
		\item Bessere Integration von \acs{wfs}/Online-Quellen: Format-Erkennung, Periodisches Abrufen, Erkennen von Änderungen, etc.
		\item SuisseID \& OpenID Authentifizierung
	\end{itemize}
\end{document}
