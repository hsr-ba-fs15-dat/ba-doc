\documentclass[class=scrbook,crop=false]{standalone}

\input{./preamble}

\begin{document}
	
	\section{Projektsitzung 11. März 2015}
	
	\begin{tabular}{ll}
		\textbf{Datum} & 11.03.2015 \\
		\textbf{Zeit} & 09:30\textendash10:30 Uhr \\
		\textbf{Ort} & \acs{ifs} an der \acs{hsr} \\
		\textbf{Anwesende} & \proff \\ & \chuf \\ & \rlif \\ & \fscf
	\end{tabular}
	

	\subsection*{Traktanden}
	\begin{itemize}
		\item Aktuellen Stand besprechen
		\item Demo Prototyp / Heroku-Deployment
		\item Abklärungen Datenschutz
	\end{itemize}
	
	\subsection*{Was wurde erreicht}
	\begin{itemize}
		\item Heroku-Deployment
		\item Prototyp CSV-Daten einlesen
		\item Prototyp REST \acs{api}
		\item Prototyp User Management
	\end{itemize}
	
	\subsection*{Beschlüsse}
	\begin{itemize}
		\item Für Benutzer-Anmeldung ist OAuth2 zu verwenden, keine eigene Benutzerverwaltung.
		\item Keine Gruppenverwaltung. Dies soll mit gemeinsam verwendeten OAuth2-IDs realisiert werden.
		\item Daten sollen als Private/Public markiert werden können.
	\end{itemize}
	
	\subsubsection*{Weiteres Vorgehen}
	\begin{itemize}
		\item Refactoring Prototypen
		\item Liste von ``Documents'' (Daten), ähnlich ckan\footnote{\url{http://ckan.org/}}.
		\item Berechtigungskonzept
		\item Plugin-System für Konverter
		\item Kurze Übersicht über ckan (Datenmodell, Features)
	\end{itemize}
	
\end{document}



