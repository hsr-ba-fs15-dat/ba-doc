\documentclass[class=scrbook,crop=false]{standalone}

\input{./preamble}

\begin{document}
	
	\section{Projektsitzung 27. Mai 2015}
	
	\begin{tabular}{ll}
		\textbf{Datum} & 27.5.2015 \\
		\textbf{Zeit} & 11:00\textendash12:00 Uhr \\
        \textbf{Ort} & \acs{ifs} an der \acs{hsr} \\
        \textbf{Anwesende} & \proff \\ & \chuf \\ & \rlif \\ & \fscf 
	\end{tabular}

	\subsection*{Traktanden}
	\begin{itemize}
		\item \acs{gdal} Versionschaos besprechen
		\item Weiteres Vorgehen besprechen
	\end{itemize}
	
	\subsection*{Was wurde erreicht}
	\begin{itemize}
		\item GeoCSV nach neuer Spezifikation implementiert
		\item Diverse Bugfixes
		\item UI Improvements
	\end{itemize}

	\subsection*{Beschlüsse}
	\begin{itemize}
		\item Wir bleiben bei \acs{gdal} 1.10.1, da diese die meisten für OpenDataHub relevanten Formate unterstützt ohne Probleme bzw. Segmentation Faults zu verursachen.
	\end{itemize}
	
	\subsubsection*{Weiteres Vorgehen}
	\begin{itemize}
		\item Last Minute Bugfixes
		\item Dokumentation / BA Dokumente
	\end{itemize}
\end{document}
