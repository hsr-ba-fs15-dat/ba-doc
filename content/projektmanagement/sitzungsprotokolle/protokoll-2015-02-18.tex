\documentclass[class=scrbook,crop=false]{standalone}

\input{./preamble}

\begin{document}
	
	\section{Projektsitzung 18. Februar 2015}
	
	\begin{tabular}{ll}
		\textbf{Datum} & 18.02.2015 \\
		\textbf{Zeit} & 13:30\textendash15:00 Uhr \\
		\textbf{Ort} & \acs{ifs} an der \acs{hsr} \\
		\textbf{Anwesende} & \proff \\ & \rlif \\ & \fscf
	\end{tabular}
	
	\subsection*{Traktanden}
	\begin{itemize}
		\item Kickoff
		\item Sachlage bezüglich Teilnahme von \chuf klären
	\end{itemize}
	
	\subsection*{Was wurde erreicht}
	\begin{itemize}
		\item Projektinfrastruktur teilweise aufgesetzt
		\begin{itemize}
			\item Slack
			\item Atlassian Jira
			\item Template für die Dokumentation mit \LaTeX
		\end{itemize}
	\end{itemize}
	
	\subsection*{Beschlüsse}
	\begin{itemize}
		\item Vorerst wird mit einer Zweierarbeit gerechnet bis die Sachlage von \chu geklärt wurde.
		\item Agiles Vorgehen mit zweiwöchigen Iterationen
		\item Titel der Arbeit lautet vorerst ``Project dat \acs{hsr}''
		\item Zunächst folgende Ziele:
		\begin{itemize}
			\item \gls{dat} Einarbeitung und Beschreibung innerhalb der Dokumentation.
			\item Integration mit ogrtools\footnote{\url{https://pypi.python.org/pypi/ogrtools}} abschätzen.
			\item Vereinheitlichung von Adressen mit dem Zielformat \gls{mopublic}.
		\end{itemize}
		\item \gls{dat} Alternativen sind zu erwähnen, jedoch ohne detaillierte Evaluation.
		\item Die Aufgabenstellung wird im \acs{hsr} Wiki\footnote{\url{http://wiki.hsr.ch/StefanKeller/SA_BA_FS15_ProjectDatHSR}} angepasst/erweitert.
	\end{itemize}
	
	\subsubsection*{Weiteres Vorgehen}
	\begin{itemize}
		\item Projektinfrastruktur aufsetzen und dokumentieren
		\item Einlesen in \gls{dat}
		\item Erste Risikoanalyse
		\item Planung des ersten Sprints
		\item Die Projektsitzungen finden bis Ostern jeweils jeden Mittwoch um 13:30 Uhr im \acs{ifs} statt.
	\end{itemize}
	
\end{document}



