\documentclass[class=scrbook,crop=false]{standalone}

\input{./preamble}

\begin{document}
	
	\section{Projektsitzung 26. März 2015}
	
	\begin{tabular}{ll}
		\textbf{Datum} & 26.03.2015 \\
		\textbf{Zeit} & 14:00\textendash15:30 Uhr \\
		\textbf{Ort} & Restaurant im Seminarhotel Spirgarten in Zürich Altstetten \\
		\textbf{Anwesende} & \proff \\ & Sepp Dorfschmid\footnotemark{} \\ & \rlif \\ & \fscf
	\end{tabular}
	\footnotetext{Autor des INTERLIS Austauschformat. Siehe auch \purl{http://www.adasys.ch} bzw. \purl{http://www.interlis.ch}}
	
	\subsection*{Traktanden}
	\begin{itemize}
		\item Aktuellen Stand besprechen
		\item Anforderungen bei Herrn Dorfschmid aufnehmen
		\item Lösungsansätze für die Datenintegration/Schematransformation diskutieren
	\end{itemize}
	
	\subsection*{Was wurde erreicht}
	\begin{itemize}
		\item Authentifizierung via Facebook und GitHub
		\item Parsen diverser Formate\footnote{GML, KML, XML, JSON, CSV, \dots} in DataFrame Objekte von Pandas\footnote{http://pandas.pydata.org/}
		\item Konversion/Formatierung der DataFrame Objekte wiederum in diverse Formate
	\end{itemize}
	
	\subsection*{Ablauf und Beschlüsse}
	\begin{itemize}
		\item Das Projektteam erläutert die Aufgabenstellung aus eigener Sicht sowie die bisherige Architektur und potenzielle Erweiterungsmöglichkeiten mit IPython um die Daten ad-hoc manipulieren zu können.
		\item Herr Dorfschmid bespricht mit dem Projektteam seine Sicht bzw. Anforderungen an eine solche Integrationsplattform und gibt sogleich Vorschläge für die Transformation bzw. Mapping der Daten
		\item Die Lösung gleicht im Prinzip einer ``SQL-View'' wobei die Daten selektiv mit einer Mapping-Konfiguration bzw. Query-Language in das gewünschte Format (View) gebracht werden.
		\item Die Sprache sollte nebst der Selektion auch Joins bzw. Unions und einige String Operationen beherrschen.
		\item Aus Sicht von Herrn Dorfschmid wären auch Geo-Funktionen sowie eine Navigation über Attribute/Felder sinnvoll. Letzteres setzt voraus, dass komplexe (geschachtelte) Objekte in einem Feld gespeichert werden können.
		\item Der Fokus wird ab jetzt, vor allem auch aufgrund der Zeitbegrenzung, auf die Transformation gesetzt. Analysemöglichkeiten der Daten und weitere Features sind nice to have.
		\item Die nächste Sitzung findet am Mittwoch 1. April um 10:00 Uhr im \acs{ifs} statt.
	\end{itemize}
	
	\subsubsection*{Weiteres Vorgehen}
	\begin{itemize}
		\item SuisseID Authentifizierung
		\item Entwurf und Implementation einer Mapping-Sprache
	\end{itemize}
\end{document}
