\documentclass[class=scrbook,crop=false]{standalone}

\input{./preamble}

\begin{document}
	
	\section{Projektsitzung 12. Mai 2015}
	
	\begin{tabular}{ll}
		\textbf{Datum} & 12.5.2015 \\
		\textbf{Zeit} & 15:00\textendash16:00 Uhr \\
        \textbf{Ort} & \acs{ifs} an der \acs{hsr} \\
        \textbf{Anwesende} & \proff \\ & \chuf \\ & \fscf \\ & (\rlif wegen Militär abwesend)
	\end{tabular}

	\subsection*{Traktanden}
	\begin{itemize}
		\item Prototyp GeoCSV besprechen
		\item (UI) Testing besprechen
		\item Screencast: Was wird erwartet?
	\end{itemize}
	
	\subsection*{Was wurde erreicht}
	\begin{itemize}
		\item GeoCSV Export/Import
		\item INTERLIS Modell-Generierung
		\item Testing, z.B. Formattieren aller Testdaten mit allen Formaten
	\end{itemize}

	\subsection*{Beschlüsse}
	\begin{itemize}
		\item CSVT-Support auch für normale CSV-Dateien
        \item Nach Verbesserung des INTERLIS Modell-Generators soll dieser beim Download von Daten im Format INTERLIS1 direkt verwendet werden. Falls das generierte Modell fehlerhaft ist, Fallback auf den bisher verwendeten Modell-losen Export.
        \item Screencast: Transformation mit Adress-Daten, anschliessed mit Geo-Daten um weitere Funktionalität zu demonstrieren.
   	\end{itemize}
	
	\subsubsection*{Weiteres Vorgehen}
	\begin{itemize}
		\item Bugfixing
		\item (UI) Testing
	\end{itemize}
\end{document}
