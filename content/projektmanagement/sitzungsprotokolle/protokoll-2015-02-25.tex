\documentclass[class=scrbook,crop=false]{standalone}

\input{./preamble}

\begin{document}
	
	\section{Projektsitzung 25. Februar 2015}
	
	\begin{tabular}{ll}
		\textbf{Datum} & 25.02.2015 \\
		\textbf{Zeit} & 13:30\textendash15:00 Uhr \\
		\textbf{Ort} & \acs{ifs} an der \acs{hsr} \\
		\textbf{Anwesende} & \proff \\ & \chuf \\ & \rlif \\ & \fscf
	\end{tabular}
	
	\subsection*{Traktanden}
	\begin{itemize}
		\item Bericht erster Erfahrungen mit \gls{dat}
		\item Projektrisiken besprechen, vor allem das Risiko ``\gls{dat}''
		\item Scope der Arbeit nochmals besprechen
		\item Erste simple User Story ``Weihnachtskartenversand'' aufnehmen
	\end{itemize}
	
	\subsection*{Was wurde erreicht}
	\begin{itemize}
		\item Projektinfrastruktur / Collaboration Plattform vollständig eingerichtet
		\item Erste Erfahrungen und Tests mit \gls{dat}
		\item Erste Risikoabschätzung
	\end{itemize}
	
	\subsection*{Beschlüsse}
	\begin{itemize}
		\item \gls{dat} hat Potenzial, steckt jedoch noch in den Kinderschuhen. Aus diesem Grund wird auf \gls{dat} verzichtet und neu ein eigener ``Data Hub'' in Form einer modernen Webapplikation umgesetzt.
		\item Der Datenhub soll die Exportformate von \purl{http://openpoimap.ch} sowie GeoPackage\footnote{\url{http://www.geopackage.org/}} und weitere gängige Formate\footnote{\acs{csv}, JSON, \dots} beherrschen.
		\item Der Datenhub soll ein mit \gls{wfs} kompatibles \acs{rest} \acs{api} bieten.
		\item Der Benutzer will beim Abruf der Daten entscheiden können, was er alles sehen will (Felder).
		\item Die nächste Sitzung findet am Mittwoch, 4. März um 09:30 Uhr statt.
	\end{itemize}
	
	\subsubsection*{Weiteres Vorgehen}
	\begin{itemize}
		\item Einführende Dokumentation über \gls{dat} (Deliverable) abschliessen und zurück auf Start.
		\item Sprint 1 wird vorzeitig beendet aufgrund der Entscheidung \gls{dat} nicht zu verwenden.
		\item Alternative Lösungsvorschläge erarbeiten.
	\end{itemize}
	
\end{document}



