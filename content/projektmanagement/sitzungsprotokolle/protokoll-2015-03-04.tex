\documentclass[class=scrbook,crop=false]{standalone}

\input{./preamble}

\begin{document}
	
	\section{Projektsitzung 04. März 2015}
	
	\begin{tabular}{ll}
		\textbf{Datum} & 04.03.2015 \\
		\textbf{Zeit} & 09:30\textendash10:30 Uhr \\
		\textbf{Ort} & \acs{ifs} an der \acs{hsr} \\
		\textbf{Anwesende} & \proff \\ & \chuf \\ & \rlif \\ & \fscf
	\end{tabular}
	

	\subsection*{Traktanden}
	\begin{itemize}
		\item Erste Lösungsvorschläge diskutieren
		\item Mockups/Wireframes
	\end{itemize}
	
	\subsection*{Was wurde erreicht}
	\begin{itemize}
		\item Kapitel dat abgeschlossen \& Infrastruktur-Rückbau
		\item Neue Infrastruktur (Python/AngularJS)
		\item Risiken überarbeitet (dat)
		\item Erste Lösungsvorschläge \& Mockups erarbeitet
	\end{itemize}
	
	\subsection*{Beschlüsse}
	\begin{itemize}
		\item Die Arbeit wird unter der \acs{mit}-Lizenz veröffentlicht
		\item Code und Lauffähigkeit hat entgegen dem Standard \acs{hsr} Reglement mehr Gewicht als die Dokumentation
		\item ``OpenDataHub'' als Name/Titel
		\item Video als Deliverable ist optional
		\item Die Applikation sollte von der Behörden akzeptiert werden. Es empfiehlt sich deshalb die \acf{saga.ch} einzuhalten.
		\item Die nächste Sitzung findet am Mittwoch, 11. März um 09:30 Uhr statt.
	\end{itemize}
	
	\subsubsection*{Weiteres Vorgehen}
	\begin{itemize}
		\item \proff liefert die verschiedenen Quellen der \gls{trobdb}
		\item User Stories und Rollen ausarbeiten
		\item Mockups überarbeiten
		\item Pipeline Prototyp auf dem Backend
		\item Persistieren der Pipeline-Konfigurationen in einer Postgres Datenbank / Testdaten
		\item Evtl. Evaluation von UI Lösungen für die Modellierung der Flows
	\end{itemize}
	
\end{document}



