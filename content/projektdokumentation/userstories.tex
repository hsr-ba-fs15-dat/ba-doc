\begin{scrumepic}[label=epic:datenanbieter]{Datenanbieter}
	Als Datenanbieter, will ich die Daten in diversen strukturierten Formaten anbieten können, sodass ich diese der Öffentlichkeit ohne zusätzlichen Aufwand bereitstellen kann.
	\xxx[Epic vervollständigen]
	\storyacceptance	
	Gegeben
	\storyitem{XY,}
	\storyitem{einen modernen Webbrowser (Internet Explorer 10 oder neuer und alle Evergreen\footnote{Browser die sich automatisch auf dem neusten Stand halten} Browser),}
	wenn XY, dann soll XY.
\end{scrumepic}

\begin{scrumstory}[label=story:dateihochladen]{Datei Hochladen Story}
	Als Datenanbieter, will ich eine Datei in einem strukturierten Format hochladen können, sodass ich diese auf der Plattform weiterverwenden kann.
	\storyacceptance	
	Gegeben
	\storyitem{Datei in einem strukturierten Format}
	wenn XY, dann soll XY.
\end{scrumstory}

\begin{scrumstory}[label=story:pfadangeben]{Web Adresse}
	Als Datenanbieter, will ich Daten in einem strukturierten Format von einemr Webadresse beziehen können, sodass ich diese auf der Plattform weiterverwenden kann.
	\storyacceptance	
	Gegeben
	\storyitem{Daten sind Online in einem strukturierten Format}
	wenn die Daten, dann soll XY.
\end{scrumstory}
\begin{scrumstory}[label=story:modulanbieten]{Modul anbieten}
	Als Datenanbieter oder Entwickler, will ich Python-Module mit einer bestimmten Schnittstelle anbieten können, um die bereitgestellten Daten umwandeln und filtern zu können.
	\storyacceptance	
	Gegeben
	\storyitem{Daten sind Online in einem strukturierten Format}
	wenn die Daten, dann soll XY.
\end{scrumstory}

\begin{scrumepic}[label=epic:datenbezug]{Datenbezug}
	Als Endbenutzer will ich die bereitgestellten Daten in einem von mir gewünschten, strukturierten Format, beziehen können.
	\xxx[Endbenutzer Epic]
	\storyacceptance	
	Gegeben
	\storyitem{XY,}
	\storyitem{einen modernen Webbrowser (Internet Explorer 10 oder neuer und alle Evergreen\footnote{Browser die sich automatisch auf dem neusten Stand halten} Browser),}
	wenn XY, dann soll XY.
\end{scrumepic}
\begin{scrumstory}[label=story:modulanbieten]{Modul anbieten}
	Als Endbenutzer will ich die bereitgestellten Daten über das Webportal mit wenigen Klicks herunterladen können um diese auf meine Computer weiter zu verwernden.
	\storyacceptance	
	Gegeben
	\storyitem{Daten sind Online in einem strukturierten Format}
	wenn die Daten, dann soll XY.
\end{scrumstory}

\begin{scrumstory}[label=story:drittentwickler]{Drittentwickler}
	Als Drittentwickler will ich die bereitgestellten Daten über eine \gls{rest} Schnittstelle beziehen können um diese in einem anderen Projekt zu verwenden.
	\storyacceptance	
	Gegeben
	\storyitem{Daten sind Online in einem strukturierten Format}
	wenn die Daten, dann soll XY.
\end{scrumstory}
