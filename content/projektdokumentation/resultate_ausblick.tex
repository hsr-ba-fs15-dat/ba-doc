\chapter{Resultate und Ausblick}

\section{Resultate}
Die Resultate der Arbeit sind in \vref{sec:tb:results} beschrieben.

\section{Weiterentwicklung} \label{sec:tb:further-dev}

\subsection{GDAL 2.0}
Diverse Treiber wurden in \gls{gdal} 2.0 massiv verbessert (\gls{interlis} 1/2) bzw. richtig implementiert (GeoPackage). Sobald ein Release vorhanden ist, sollte ein Upgrade in Betracht gezogen werden.

\subsection{Native Unterstützung für via ogr2ogr implementierte Formate}
Die Unterstützung für die Formate \gls{interlis} 1, GML und WFS (nur Parser) ist via ogr2ogr implementiert. Da ogr2ogr auf Datei-Ebene operiert, müssen die Daten über ein Zwischenformat konvertiert werden.

Für den Formatter sieht das wie folgt aus:
$$ \text{DataFrame in ODH} \to \text{Zwischenformat} \to \text{ogr2ogr} \to \text{Zielformat} $$

Analog für den Parser. Beide Konvertierungs-Schritte sind potentiell fehlerbehaftet, ausserdem sind nicht alle ogr2ogr-Treiber von gleicher Qualität.

Daher wäre es wünschenswert, diese Formate mit nativen Parsern und Formattern zu unterstützen. 

\subsection{Ausbau der Webapplikation}
Nach Untersuchung der bereits vorhandenen Produkte wurde entschieden, die Webapplikation minimal zu halten. Die Konvertierung und Transformation von Daten hatte Priorität.

Wir sehen folgende Optionen:
\begin{itemize}
\item Ausbau der verwalteten Metadaten
\item Veränderung der Files, die zu einem Package gehören
\item Diskussion von Daten/Transformationen, evtl. Issue-Tracker
\item Integration von CKAN
\end{itemize}

\subsection{Unterstützung von verschiedenen Encoding-Optionen}
Sofern vom Datei-Format nichts anderes vorgegeben wird, lesen die implementierten Parser allen Input mit UTF-8 als Encoding. Gerade bei Formaten wie CSV, welche von Benutzern direkt erstellt werden können, z.B. mit Excel, ist dies nicht immer optimal.

Wir sehen folgende Optionen:
\begin{itemize}
\item Dem Benutzer eine Angabe des verwendeten Encodings ermöglichen
\item Eine Encoding-Detection durchführen, z.B. mit chardet\footnote{\url{https://pypi.python.org/pypi/chardet}}. Dies funktioniert in vielen Fällen einigermassen gut, ist jedoch nicht zuverlässig.
\end{itemize}

Am besten wäre wohl eine Kombination der beiden Optionen.

\subsection{Zusätzliche ODHQL-Funktionen}
Bisher sind zwar diverse Funktionen implementiert, gerade im Geo-Bereich gibt es jedoch noch einiges an Erweiterungspotential.

\subsection{Datentypen}
ODHQL unterstützt bisher folgende Datentypen: Integer (32 Bit), SmallInt (16 Bit), BigInt (64 Bit), Float (64 Bit), DateTime (64 Bit), Boolean, Text und Geometry. (Date-/Time-)Interval wird theoretisch unterstützt, es gibt jedoch aktuell keine Funktionen welche diesen Typ verwenden.

Mögliche Erweiterungen:
\begin{itemize}
\item Wertebereiche für bereits bestehende Datentypen, z.B. Längenbeschränkung für Text, oder Boundaries für Geometrien
\item Subqueries bzw. Nested Tables
\item Baumstrukturen, Key-Value-Pairs
\end{itemize}

Ausserdem sollte untersucht werden, ob Geodaten in allen Fällen korrekt behandelt werden.

\subsection{Breitere Formatunterstützung}
Nach Bedarf können weitere Formate implementiert werden.

\subsection{ODH als WFS-Server}
ODH unterstützt WFS als Daten-Quelle. Die Idee ist, dass eine API implementiert wird, welche eine Datenquelle oder WFS-Url und optional eine ODHQL-Abfrage entgegen nimmt, die Transformation durchführt und anschliessend das Resultat als WFS anbietet.

\subsection{Performanz bei grossen Datenmengen}
Im Rahmen der Arbeit wurden Dokumente mit Grössen bis ca. 100 MB (ca. 1 Mio. Records) mit guten Resultaten getestet. Im Gespräch mit Herrn Dorfschmid wurde festgestellt, dass dies für viele Anwendungen ausreicht.

Falls massiv grössere Datenmengen verarbeitet werden müssen ist die aktuelle Architektur aber suboptimal und stark limitiert durch den verfügbaren Speicher. In diesem Falle sollte evtl. auf eine streambasierte Version gewechselt werden.

Zu beachten ist, dass diverse verwendete Bibliotheken (unter anderem auch \gls{gdal}) nicht streamorientiert arbeiten.

\subsection{Unterstützung für weitere Datenstrukturen}
Die aktuelle Version von \acs{odh} unterstützt nur tabellarische Daten. Dies könnte um Unterstützung für baumartige Daten, wie zum Beispiel XML, erweitert werden. 

\subsection{Unterstützung für weitere Transformationsarten}
Ogr2ogr ermöglicht es, Transformationen per SQL in einer Datenbank auszuführen. Dies könnte auch in ODH implementiert werden, würde jedoch zu Performance-Verlusten führen.

\subsection{Bereinigung des REST API}
Das \gls{rest} \acs{api} funktioniert, beinhaltet jedoch einige Inkonsistenzen bzw. Unschönheiten. 

Ein Beispiel:
Bei Anfragen auf \url{/api/v1/fileGroups/} wird das zur File Group gehörende Dokument zurückgeliefert.
\url{/api/v1/document/<id>/fileGroups/} liefert alle zu einem Dokument gehörenden File Groups. Auch hier enthält die Antwort das Dokument, da derselbe Serializer verwendet wird. Dies führt jedoch dazu, dass bei Dokumenten mit mehr als einer File Group die Dokument-Informationen mehrfach in der Antwort enthalten sind.

Stattdessen sollte in dieser Situation nur die Dokument-Url geliefert werden.

\subsection{Templates in neuen Transformationen verwenden können}
Anstatt ein Template zuerst als Transformation abzuspeichern, könnte es sinnvoll sein, ein Template direkt in einer neuen Transformation verwenden zu können.

\subsection{In Templates Spaltennamen überschreiben}
Im Moment können im OpenDataHub die Tabellennamen generalisiert werden. Wenn ein Template andere Spaltennamen als das referenzierte Dokument verwendet, können diese nicht automatisch ausgetauscht werden.

\subsection{Kategorien und Tags für Dokumente und Transformationen}
In einer weiteren Version könnten Dokumente und Transformationen mit Tags und Kategorien versehen werden. Dies würde die Übersichtlichkeit erhöhen.

\subsection{Benutzer-Gruppen mit unterschiedlichen Rechten}
Zu Beginn der Arbeit wurde zusammen mit dem Betreuer entschieden, dass die Benutzer-Verwaltung minimal gehalten werden soll. Insbesondere sollte kein Aufwand zur Administration der Benutzer anfallen.

In Zukunft könnte es wünschenswert sein, Benutzergruppen mit unterschiedlichen Rechten einzuführen, z.B. eine Unterscheidung zwischen normalen Nutzern und Administratoren.