\section{Format-Unterstützung}
\subsection{Unterstützte Formate}
\cref{tab:pd:formats} beschreibt die implementierten Formate.

\mytable{lX}{
  \textbf{Format} & \textbf{Implementiert via}\\
  \midrule
  CSV (inkl. GeoCSV) & (Geo)Pandas \\
  Excel (xls, xlsx) & Pandas \\
  GML & ogr2ogr \\
  GeoJSON & GeoPandas \\
  GeoPackage\footnotemark{} & GeoPandas \\
  Interlis 1 & ogr2ogr \\
  Interlis Modell (ili) & custom \\
  JSON & Pandas \\
  KML  & fastkml, custom \\
  ESRI Shapefile & GeoPandas \\
  WFS (nur Parser) & ogr2ogr \\
  XML & custom \\
}{Beschreibung der implementierten Formate}{pd:formats}
\footnotetext{Deaktiviert, siehe \cref{sec:pd:format-gdal-problems}}

GeoPandas verwendet (via Fiona\footnote{\url{https://pypi.python.org/pypi/Fiona}}) ebenfalls \gls{gdal}.

\subsection{Probleme mit GDAL}\label{sec:pd:format-gdal-problems}
Für einen grossen Teil der Formate wird \gls{gdal} verwendet, entweder via GeoPandas/Fiona oder direkt per ogr2ogr. 

Folgende \gls{gdal}-Versionen stehen zum aktuellen Zeitpunkt zur Verfügung:
\begin{itemize}
\item 1.10.1
\item 1.11.0 - 1.11.2
\item 2.0.0beta1
\end{itemize}

Die Entwicklungsumgebung bestand aus einer virtuellen Maschine mit Ubuntu 14.04 LTS, für welches ein \gls{gdal}-Package in Version 1.10.1 vorhanden ist. \proff wünschte jedoch, dass wenn möglich Unterstützung für das relativ neue Format GeoPackage eingebaut wird. \gls{gdal} 1.10.1 unterstützt GeoPackage nicht. Im Versuch, diese Unterstützung einzubauen wurde \gls{gdal} selbst kompiliert. Dem Betreuer waren Probleme mit \gls{gdal} 1.11.2 bekannt, weswegen 1.11.1 gewählt wurde.

In Tests wurden folgende Probleme gefunden:
\begin{itemize}
\item In \gls{gdal} 1.11.0 bis 2.0.0beta1 führt der Versuch, nach Interlis 1 zu konvertieren, zu einem \gls{segfault}. Das Problem wurde vom Betreuer an den Treiber-Entwickler weiter gemeldet. In aktuellen nightly-Versionen soll dies behoben sein - dies wurde nicht verifiziert.
\item Der GeoPackage-Treiber wurde in \gls{gdal} 1.11.0 hinzugefügt. Leider kann er in dieser Version nur als unausgereift bezeichnet werden:
  \begin{itemize}
  \item Layer-Namen werden ohne Bereinigung oder Maskierung als Tabellen-Namen verwendet, was zu Exceptions seitens SQLite führt.
  \item Wenn die Ursprungs-Daten bereits eine Spalte namens ``fid'' beinhalten, bricht die Konvertierung ab wegen doppelter Spalten-Namen, da der Treiber diese Spalte selbst nochmals hinzu generiert.
  \item Reine Daten-Tabellen werden nicht unterstützt. Diese sind zwar nicht Bestandteil des reinen GeoPackage-Standards, können aber mit Extended GeoPackage realisiert werden. Dies ist in \gls{gdal} erst ab Version 2.0.0 möglich.
  \end{itemize}
\end{itemize}

GeoPackage-Support wurde in \gls{gdal} 2.0.0 massiv verbessert. Diese Version ist bei Abschluss der Arbeit jedoch noch in der Beta-Phase, weshalb diverse unserer Abhängigkeiten dazu noch nicht kompatibel sind.

\begin{decision}[label=dec:pd:gdal-version]{GDAL-Version}
Zum Abgabe-Zeitpunkt sind folgende \gls{gdal}-Versionen verfügbar: 1.10.1, 1.11.0 - .2 und 2.0.0beta1.

Wegen Problemen mit mehreren Treibern entfallen die Versionen 1.11.x.

Version 2.0.0beta1 entfällt wegen mangelnder Unterstützung in unseren Abhängigkeiten.

Daher wird vorerst \gls{gdal} 1.10.1 eingesetzt.
\end{decision}

\subsection{Interlis 2}
\begin{decision}[label=dec:pd:no_interlis2]{Kein Interlis 2 Support}
Aufgrund von Problemen mit dem Format-Support in ogr2ogr wurde in Absprache mit dem Betreuer entschieden, Unterstützung für Interlis 2 nicht zu implementieren. Siehe dazu das Sitzungsprotokoll vom 15.04.2015.
\end{decision}
\subsection{Erweiterung}
Die Unterstützung für neue Formate kann relativ einfach hinzugefügt werden. Dazu müssen Subklassen für folgende Python-Klassen erstellt werden:

\begin{description}
\item[hub.formats.core.Format] Dient zur Identifikation des Formats von Datenquellen (z.B. Dateien)
\item[hub.formats.core.Formatter] Erhält DataFrames und produziert Dateien, welche der Benutzer herunterladen kann.
\item[hub.formats.core.Parser] Erhält Daten (aus Dateien, WFS, ...) und produziert DataFrames, welche von ODH weiterverwendet werden können.
\end{description}

Für die Implementation via ogr2ogr stehen im Modul hub.formats.geobase Hilfsklassen zur Verfügung.

Durch Vererbung und die verwendete Python-Metaklasse wird das neue Format automatisch registriert und kann sofort verwendet werden.

Ein Beispiel ist in \cref{src:pd:format-example} zu sehen. Darin wird die Unterstützung für ESRI Shapefile implementiert. Der Formatter verwendet hierzu die Hilfsklasse GeoFormatterBase, der Parser jedoch direkt GeoPandas.

\begin{srclst}[label=src:pd:format-example]{python}{Beispiel-Implementation der Format-Unterstützung für ESRI Shapefile}
import geopandas as gp
import os

from hub.formats import Format, Parser
from hub.formats.geobase import GeoFormatterBase
from hub.utils import ogr2ogr

if 'ESRI Shapefile' in ogr2ogr.SUPPORTED_DRIVERS:
    class Shapefile(Format):
        label = 'ESRI Shapefile'
        ogr_format = ogr2ogr.SHP

        description = """
        Ein ursprünglich für die Firma ESRI entwickeltes Format für Geodaten.
        """

        extension = 'shp'

        @classmethod
        def is_format(cls, file, *args, **kwargs):
            return file.extension == 'shp'

    class ShapefileFormatter(GeoFormatterBase):
        targets = Shapefile,

        supported_types = {'Point', 'LineString', 'LinearRing', 'Polygon', 'MultiPoint'}

        @classmethod
        def format(cls, dfs, name, format, *args, **kwargs):
            return super(ShapefileFormatter, cls).format(dfs, name, format, 'ESRI Shapefile', 'shp', *args, **kwargs)

    class ShapefileParser(Parser):
        accepts = Shapefile,

        @classmethod
        def parse(cls, file, format, *args, **kwargs):
            with file.file_group.on_filesystem() as temp_dir:
                return gp.read_file(os.path.join(temp_dir, file.name), driver='ESRI Shapefile')
\end{srclst}