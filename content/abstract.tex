\begin{comment}
2.1.2 Abstract
Ein Abstract ist eine rein textuelle kurze Zusammenfassung der Arbeit. Der Abstract ist für die Recherche in grossen Dokumentensammlungen geeignet. Er umfasst nie mehr als eine Seite, typisch sogar nur etwa 200 Worte (etwa 20 Zeilen).
Der Begriff ‚Kurzfassung’ ist zuwenig genau definiert; er soll wenn möglich vermieden werden.
\end{comment}


\phantomsection
\addcontentsline{toc}{chapter}{Abstract}
\chapter*{Abstract}
In der Schweizer Geo-Szene wird eine Vielfalt an Geo-Software eingesetzt, welche eine grosse Anzahl Formate verwenden. Dies führt zu Problemen beim Datenaustausch. Um diese Probleme zu mildern wurde ein Prototyp für eine Konversions- und Transformationsplattform erstellt. Benutzer können Daten anbieten und beziehen, weitgehend ohne sich um Dateiformate kümmern zu müssen. Experten können anhand dieser Daten Transformationen erstellen, welche den Nutzern wiederum als Datenquellen dienen.

Als Programmiersprache wurde Python 2.7 verwendet. Die Format-Konvertierung basiert vorwiegend auf Pandas und GDAL. Zur Transformation wurde eine eigene, stark an SQL angelehnte Sprache implementiert.
\glsresetall

