\begin{comment}
2.1.2 Abstract
Ein Abstract ist eine rein textuelle kurze Zusammenfassung der Arbeit. Der Abstract ist für die Recherche in grossen Dokumentensammlungen geeignet. Er umfasst nie mehr als eine Seite, typisch sogar nur etwa 200 Worte (etwa 20 Zeilen).
Der Begriff ‚Kurzfassung’ ist zuwenig genau definiert; er soll wenn möglich vermieden werden.
\end{comment}


\phantomsection
\addcontentsline{toc}{chapter}{Abstract}
\chapter*{Abstract}

Im digitalen Zeitalter spielen Daten jeglicher Art eine zentrale Rolle für die reibungslose Zusammenarbeit zwischen diversen Organisationen und deren Applikationen. Diese Daten kommen in einer Vielzahl an Formaten und unterschiedlichen Schemas daher, welche erst durch einen komplexen und oftmals individuellen Transformationsprozess zu eigenen Zwecken wiederverwendet werden können. Ziel dieser Arbeit ist die Abstrahierung von Dateiformat und Schema-Transformation, um einen Datenaustausch zwischen diversen Parteien mit Hilfe einer HTML5-Webapplikation zu vereinfachen \textendash\ OpenDataHub.

\medskip
Nach Evaluation von vorgegebenen (\purl{http://dat-data.com}) sowie weiteren Datenaustausch-Plattformen wurde ein modulares und erweiterbares Konzept zur Konversion diverser Dateiformate sowie die Transformation mittels einer eigenen, an SQL angelehnten \acl{dsl} mit dem Namen ``OpenDataHub Query Language'' (ODHQL) entworfen und umgesetzt.

\medskip
Das Resultat ist eine moderne, mit Python 2.7, Django Framework und AngularJS umgesetzte HTML5 Webapplikation, mit der Daten diverser Formate öffentlich geteilt und durch Experten mit ODHQL-Kenntnissen transformiert werden können. Diese lassen sich dann wiederum zur Weiterverwendung in einem beliebigen Dateiformat beziehen.

\medskip
Die Erkenntnisse und das Resultat dieser Arbeit bestätigen die grundsätzliche Machbarkeit einer solchen Konversionsplattform und können als Grundlage für eine komplexere Lösung wiederverwendet werden.

% In der Schweizer Geo-Szene wird eine Vielfalt an Geo-Software eingesetzt, welche eine grosse Anzahl Formate verwenden. Dies führt zu Problemen beim Datenaustausch. Um diese Probleme zu mildern wurde ein Prototyp für eine Konversions- und Transformationsplattform erstellt. Benutzer können Daten anbieten und beziehen, weitgehend ohne sich um Dateiformate kümmern zu müssen. Experten können anhand dieser Daten Transformationen erstellen, welche den Nutzern wiederum als Datenquellen dienen.

% Als Programmiersprache wurde Python 2.7 verwendet. Die Format-Konvertierung basiert vorwiegend auf Pandas und GDAL. Zur Transformation wurde eine eigene, stark an SQL angelehnte Sprache implementiert.
\glsresetall