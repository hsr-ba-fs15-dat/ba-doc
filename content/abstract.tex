\begin{comment}
2.1.2 Abstract
Ein Abstract ist eine rein textuelle kurze Zusammenfassung der Arbeit. Der Abstract ist für die Recherche in grossen Dokumentensammlungen geeignet. Er umfasst nie mehr als eine Seite, typisch sogar nur etwa 200 Worte (etwa 20 Zeilen).
Der Begriff ‚Kurzfassung’ ist zuwenig genau definiert; er soll wenn möglich vermieden werden.
\end{comment}


\phantomsection
\addcontentsline{toc}{chapter}{Abstract}
\chapter*{Abstract}

Im digitalen Zeitalter spielen Daten jeglicher Art eine zentrale Rolle für die reibungslose Zusammenarbeit zwischen diversen Organisationen und deren Applikationen. Diese Daten kommen in einer Vielzahl an Formaten und unterschiedlichen Schemata daher, welche erst durch einen komplexen und oftmals individuellen Transformationsprozess zu eigenen Zwecken wiederverwendet werden können. Ziel dieser Arbeit ist die Abstrahierung von Dateiformat und Schema-Transformation, um einen Datenaustausch zwischen diversen Parteien mit Hilfe einer Webapplikation zu vereinfachen \textendash\ \href{http://beta.opendatahub.ch/}{OpenDataHub.ch}.

\medskip
Nach Evaluation von vorgegebenen (\purl{http://dat-data.com}) sowie weiteren Datenaustausch-Plattformen wurde ein modulares und erweiterbares Konzept zur Konversion diverser Dateiformate sowie die Transformation mittels einer eigenen \gls{dsl} mit dem Namen ``\acl{odhql}'' (\acs{odhql}) entworfen und umgesetzt. Die \acs{odhql} wurde aufgrund des bereits vorhandenen \acs{sql} Know-hows vieler Entwickler als Subset dessen umgesetzt. Die funktionalen Anforderungen an die Plattform sowie die \acs{odhql} wurde durch die Umsetzung zweier Anwendungsfälle getrieben: Die Integration von Postadressen und Verkehrshindernisse für die \gls{trobdb}.

\medskip
Das Resultat ist eine moderne, mit Python 2.7, Django Framework und AngularJS umgesetzte HTML5-Webapplikation, mit der Daten diverser Formate öffentlich geteilt und durch Experten mit \acs{odhql}-Kenntnissen transformiert werden können. Diese lassen sich dann wiederum zur Weiterverwendung in einem beliebigen Dateiformat beziehen.

\medskip
Die Erkenntnisse und das Resultat dieser Arbeit bestätigen die Machbarkeit einer solchen Konversions- und Transformationsplattform und können als Grundlage für weitere Entwicklungen verwendet werden.

\bigskip
Webseite: \purl{http://beta.opendatahub.ch}


\glsresetall